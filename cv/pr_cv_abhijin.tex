\documentclass[margin,10pt]{res} % Use the res.cls style, the font size can be changed to 11pt or 12pt here
%
%%%%%%%%%%%%%%%%%%%%%%%%%%%%%%%%%%%%
%% Common preamble
%%%%%%%%%%%%%%%%%%%%%%%%%%%%%%%%%%%%
% PAGE
%% \usepackage{fullpage}
% FONTS
\usepackage{url} % enhanced version of computer modern
\urlstyle{same}
\usepackage{lmodern} % enhanced version of computer modern
\usepackage[T1]{fontenc} % for hyphenated characters
\usepackage{amssymb}
\usepackage{mathtools} % contains amsmath which comes with align
\usepackage{amsthm}
\usepackage{microtype} % some compression
%%
%% REFERENCING DOES NOT WORK WITH BIBENTRY PACKAGE
\usepackage{natbib}
\usepackage[colorlinks=false,pdfborder={0 0
 0},citecolor=magenta,linkcolor=magenta]{hyperref}

\usepackage{bibentry}
% LISTS
\usepackage{enumerate}
% ROMAN NUMERALS
\makeatletter
\newcommand{\rmnum}[1]{\romannumeral #1}
\newcommand{\Rmnum}[1]{\expandafter\@slowromancap\romannumeral #1@}
\makeatother
%%%%%%%%%%%%%%%%%%%%%%%%%%%%%%%%%%%%
%% preamble ends
%% from now on, draft specific
%%%%%%%%%%%%%%%%%%%%%%%%%%%%%%%%%%%%
\setlength{\textwidth}{5.1in} % Text width of the document
\usepackage{etoolbox}
\usepackage{multicol}
%% adding footer
\usepackage{fancyhdr}
\usepackage{xcolor}
%%\fancypagestyle{mypagestyle}{%
%%\fancyhf{}%
%%\fancyfoot[C]{\textcolor{red}{\thepage}}%
%%}
\rfoot{\textcolor{gray}{Updated on \today}}
\fancyhead{}
\renewcommand{\headrulewidth}{0pt}
\pagenumbering{gobble}
\newtoggle{visa}
\newtoggle{compact}
%\toggletrue{visa} % comment this if you want normal resume
\toggletrue{compact} % comment this if you want normal resume

\begin{document}
\thispagestyle{fancy}
%----------------------------------------------------------------------------------------
%	NAME AND ADDRESS SECTION
%----------------------------------------------------------------------------------------

%\moveleft.5\hoffset\centerline{\large\bf Abhijin Adiga\hfill} % Your name at the top
\moveleft1.0\hoffset\centerline{\large\bf Abhijin Adiga\hfill} % Your name at the top
 
\moveleft\hoffset\vbox{\hrule width\resumewidth height 1pt}\smallskip % Horizontal line after name; adjust line thickness by changing the '1pt'
 
%----------------------------------------------------------------------------------------

\begin{resume}

%----------------------------------------------------------------------------------------
%	OBJECTIVE SECTION
%----------------------------------------------------------------------------------------
%%  
%% \section{OBJECTIVE}  
%% 
%% A position in the field of computers with special interests in business applications programming, information processing, and management systems. 
%% 
%----------------------------------------------------------------------------------------
%	EDUCATION SECTION
%----------------------------------------------------------------------------------------
\section{\textnormal{\textsc{Contact Information}}}
Network Dynamics and Simulation Science Laboratory\\
Biocomplexity Institute of Virginia Tech\hfill \texttt{email:}
abhijin@vbi.vt.edu\hspace{3mm}\mbox{}\\
1015 Life Science Circle\hfill \texttt{phone:}
\texttt{+}1 540 204 6679\hspace{7mm}\mbox{}\\
Blacksburg, VA 24061, USA \\
\url{https://www.bi.vt.edu/ndssl/people/Abhijin-Adiga}
%%%%
\section{\textnormal{\textsc{Research Interests}}}
{\bf Broad areas:} 
Combinatorics;
Design \& analysis of algorithms;
Game theory; 
Computational modeling; and
Network science\\
{\bf Application areas:} Computational epidemiology; Modeling and analysis
of large socio-technical systems; Invasive species modeling
%%
\section{\textnormal{\textsc{Current\,\,\,\, Position}}}
\textbf{Research Assistant Professor} \hfill Jul 2016 -- present\\
Network Dynamics and Simulation Science Laboratory\\
Biocomplexity Institute of Virginia Tech
%%%%
\section{\textnormal{\textsc{Education}}}
\textbf{PhD:} \hfill August 2006 -- March 2011\\
Dept. of Computer Science and Automation\\
Indian Institute of Science, Bangalore, India\\ 
Adviser: {Dr. L Sunil Chandran}\\
{Thesis}: On Dimensional Parameters of Graphs and Posets\\
~\\
\noindent\textbf{Master of Science (Engg):} \hfill August 2001 -- August 2003\\
Dept. of Electrical Engineering,\\
Indian Institute of Science, Bangalore, India\\
Adviser: Dr. K R Ramakrishnan\\
{Thesis:} Cyclic Multirate Structures Based on Symmetric-Periodic
Sequences\\
~\\
\noindent\textbf{Bachelor of Engineering:} \hfill 1996 -- 2000\\
Bangalore University (B.M.S. College of Engineering)\\
Telecommunication Engineering
%%%%%%
\section{\textnormal{\textsc{Work\,\,\,\, Experience}}} 
Senior Research Associate \hfill May 2014 -- Jun 2016\\
Postdoctoral Associate \hfill October 2011 -- May 2014\\
Network Dynamics and Simulation Science Laboratory\\
Biocomplexity Institute of Virginia Tech\\
~~~\\
Research Associate \hfill March 2011 -- September 2011\\
Dept. of Computer Science and Automation, IISc \\
~~~\\
{Beceem Communications Pvt Ltd}\hfill August 2004 -- July 2006\\
I was involved in the design of algorithms for WiMax (802.16) \\
~~~\\
Project Associate \hfill October 2003 -- April 2004\\ 
Dept. of Electrical Engineering, IISc \\
%% Worked under Dr. P S Sastry on temporal data-mining \\
~~~\\
Project Associate\hfill October 2000 -- August 2001\\ 
Dept. of Electrical Engineering, IISc \\
%% Worked under Dr. K R Ramakrishnan on source \& channel coding and Advanced
%% Encryption Standard (Rijndael)
%%%%%%%%%%%%
\section{\textnormal{\textsc{Awards}}}
\begin{enumerate}[$\circ$]
   \item ``DSFEW Early Career Researchers Travel Fund'', KDD 2016.
\item ``Honorable Mention For Outstanding Novelty of Research
Question'' award for the paper ``Sensitivity of Diffusion Dynamics to
Network Uncertainty'' in AAAI'13.
\item Infosys Fellow: awarded to select
PhD candidates in IISc by Infosys Technologies Ltd. 
\item Secured All India Rank of 34 in GATE 2000 (EC), a national level
entrance exam for post graduate studies.
\item Ranked 7th in Bangalore University in Telecommunication Engg. (Year 2000).
\end{enumerate}
%%%%%%
\section{\textnormal{\textsc{Grants}}}
USAID IPM Innovation Labs: \emph{A High-resolution Interaction Based Approach to Modeling the Spread
of Agricultural Invasive Species}\\
Amount:~\$1,000,000 (\$600,000 for Virginia Tech); Duration: Oct'15--Nov'19\\
Role: PI
%%
\section{\textnormal{\textsc{News}}}
\begin{enumerate}[$\circ$]
\item
\href{https://vtnews.vt.edu/articles/2016/05/outreach-oiredinnovationlabawards.html}{Virginia Tech awards more than \$11 million to help feed people in
developing countries}
\item
\href{https://www.bi.vt.edu/ndssl/news/virginia-tech-research-team-fights-the-spread-of-invasive-pests}{Virginia
Tech Research Team Fights the Spread of Invasive Pests}
\end{enumerate}
%%
\section{\textnormal{\textsc{Publications}}} 
\nobibliography*
\noindent\textbf{Journal articles}\\
\begin{enumerate}[1.]%%[$\circ$]
\item \bibentry{zhang2016near}
\item \bibentry{adiga2016activity}
\item \bibentry{adiga2016inferring}
\item \bibentry{adiga2014sensitivity}
\item \bibentry{adiga2014representing}
\item \bibentry{adiga2014constant}
\item \bibentry{wu2014limit}
\item \bibentry{adiga2014lower}
\item \bibentry{adiga2014cubicity}
\item \bibentry{adiga2011boxicity}
\item \bibentry{adiga2010hardness}
\item \bibentry{adiga2010cubicity}
\item \bibentry{adiga2009cubicityThreshold}
\end{enumerate}
\noindent\textbf{Conference}\\
\begin{enumerate}[1.]%%[$\circ$]
\item \bibentry{adiga2016delay}
\item \bibentry{adiga2016temporal}
\item \bibentry{adiga2015csd}
\item \bibentry{zhang2015group}
\item \bibentry{adiga2015inferring}
\item \bibentry{adiga2015network}
\item \bibentry{adiga2015effect}
\item \bibentry{saha2015spectrum}
\item \bibentry{saha2014game}
\item \bibentry{adiga2013modelingurban}
\item \bibentry{adiga2013subgraph}
\item \bibentry{adiga2013robust}
\item \bibentry{adiga2013sensitivity}
\item \bibentry{adiga2012representing}
\item \bibentry{adiga2012polynomial}
\item \bibentry{adiga2011cubicity}
\item \bibentry{adiga2011constant}
\item \bibentry{adiga2010parameterized}
\item \bibentry{adiga2010boxicity}
\item \bibentry{adiga2009cubicity}
\item \bibentry{adiga2003design}
\end{enumerate}
\noindent\textbf{Workshop/poster}\\
\begin{enumerate}[1.]
\item \bibentry{venkat2016kdd}
\item \bibentry{adiga2015bme}
\item \bibentry{adiga2015synthetic}
\item \bibentry{adiga2012route}
\end{enumerate}
\noindent\textbf{Under review/Technical reports}\\
\begin{enumerate}[1.]
\item \bibentry{mortveit2015synthetic}
\item \bibentry{adiga2015sublinear}
\item \bibentry{adiga2015interaction}
\end{enumerate}
%%
\section{\textnormal{\textsc{Invited talks}}} 
\begin{enumerate}[1.]
   \item Monitoring spread of T. absoluta using a multi-layered network
   based modeling framework, \emph{Symposium on Global Spread and Management
   of the South American Tomato Leafminer, Tuta absoluta. International
Congress of Entomology}, 2016
\item How to stop an epidemic?  Games and near-optimal algorithms,
\emph{Dept. of Computer Science and Automation, Indian Institute of
Science}, 2014
\item Sensitivity of Dynamical Properties to Network Uncertainty,
\emph{Dept. of Computer Science and Automation, Indian Institute of
Science}, 2013
\end{enumerate}
%%
\section{\textnormal{\textsc{Students \\Advising}}}
\noindent\textbf{PhD}\\
Sichao Wu (Thesis adviser: Henning Mortveit)\vspace{.2cm}\\
\noindent\textbf{Undergraduates}\\
Bryan Kaperick (Spring'2016--)\\
Amleshwar Kumar (Intern: Fall'2016)
%%
\section{\textnormal{\textsc{Talks}}}
\begin{enumerate}[$\circ$]
   \item ICE, 2016, Monitoring the spread of T. absoluta using a
   multi�layered network based modeling framework
   \item (invited) IISc, 2014, How to stop an epidemic? Games and near-optimal
   algorithms
   \item ICDM, 2013, Subgraph Enumeration in Dynamic Graphs
   \item AAAI, 2013, Sensitivity of Diffusion Dynamics to Network Uncertainty
   \item (invited) IISc, 2013, Sensitivity of dynamic properties to network
   uncertainty
   \item SoCG, 2012, Representing a cubic graph as the intersection graph
   of axis-parallel boxes in three dimensions
   \item (invited) Bergen, 2009, Lower Bounds for Boxicity
   \item Eurocomb, 2009, Cubicity of Interval Graphs and the Claw Number
\end{enumerate}
%%
\section{\textnormal{PhD \textsc{Committees}}}
Sudip Saha (NDSSL, VBI, Virginia Tech)
%%
\section{\textnormal{\textsc{Professional Service}}}
\noindent\textbf{Technical Program Committee member}\\ 
\begin{enumerate}[$\circ$]
\item CSoNet~(2016) 
\item CONECCT~(2015)
\item SDM-Networks~(2015) 
\item SIAMNS~(2015)
\end{enumerate}
\noindent\textbf{Reviewer}\\ 
\noindent
Order~(2015), Australasian Journal of Combinatorics~(2015),
Algorithmica~(2014), Journal of Autonomous Agents and Multi-Agent
Systems~(2013), 
Information Processing Letters~(2012), 
Graphs and
Combinatorics~(2011), CATS~(2011)\smallskip\\
\noindent\textbf{Miscellaneous}\\ 
\begin{enumerate}[$\circ$]
\item Member of graduate students admission team in NDSSL for the Fall'16
admissions
\item Organized NDSSL graduate seminar series for the academic year 2013-2014
\end{enumerate}

%%%%%%%%%%%%%%%%%%%%%%%%%%%%%%%%%%%%%%%%%%%%%%%%%%%%%%%%%%%%%%%%%%%%%%%%%%%%%%%%%%
% For visa
%%%%%%%%%%%%%%%%%%%%%%%%%%%%%%%%%%%%%%%%%%%%%%%%%%%%%%%%%%%%%%%%%%%%%%%%%%%%%%%%%%
\iftoggle{visa}{
\newpage
\section{\large{Job description}:}
\vspace{.8cm}
%%
\section{\textnormal{{Current job}}}\label{sec:currentJob}
My work is in the area of mathematical foundations of big data and network science with
focus on graph dynamical systems, algorithm design, graph theoretical
analysis and modeling which have diverse applications in
technological, social and biological systems. Through collaborative
efforts, I analyze and develop algorithms for solving
combinatorial problems that arise out of the study of such systems.
%%
\section{\textnormal{{Postdoctoral associate}}}\label{sec:postdoc}
(October 2011 -- May 2014) Job description is same as current
job.\smallskip
%%
\section{\textnormal{{Research associate}}}
(March 2011 to October 2011) I continued my PhD work.\smallskip
%%
\section{\textnormal{PhD}}
(August 2006 to March 2011)
The work of this thesis falls into the broad area of theoretical computer
science, and in particular, concerns graph theory, combinatorics and
algorithms. We study some geometric representations of graphs and
dimensional properties of graphs and posets. The main objective is
to represent a graph using certain geometric objects. Many problems
on graphs which are otherwise computationally hard to solve can be
solved efficiently if the graph in consideration has a simple geometric
representation. In the parlance of theoretical computer science, many
NP-hard problems may be either solved in polynomial time or at least
have good approximation algorithms when a graph has a low dimensional
geometric representation. Besides, the representations which we consider
have applications in various other scientific disciplines: ecology,
social sciences, psychology, telecommunication and operations research.
Our contribution to this area are as follows:
(1)~We have developed efficient algorithms to
obtain such representations for certain specific classes of graphs.
(2)~We have complexity results which show that for some classes of graphs, it
is computationally hard to obtain such a representation.
(3)~We give a general framework relating different representations.
%%
\section{\textnormal{Beceem Communications}}\label{sec:beceem}
(August 2004 -- July 2006)
I was part of the algorithm development and simulation group. My
work involved the design and implementation of algorithms for WiMax
(802.16), a wireless broadband standard. I worked on (1)~implementing
and maintaining the error control coding module in the simulator, 
%I had
%to implement the turbo encoder and decoder, a class of high-performance
%forward error correction (FEC) codes. The language in which the simulator
%was implemented was MATLAB.
(2)~The creation of an interface between
the simulator (coded in MATLAB) and the chip (coded using Verilog), 
%I was involved in this project right from its initiation. This interface
%provided the vital link between the two modules and enabled the
%verification.
(3)~Running simulations by
varying parameters such as type of codec, channel, etc.
%%
\section{\textnormal{Project associate}}\label{sec:dataMining}
(October 2003 to April 2004)
The work was in the area of temporal data-mining. 
%We were given access
%to months of data from a manufacturing line of a factory. Each time
%the manufacturing process stopped, the time and the reason behind
%the stoppage was logged into the system. The reasons could be anything
%ranging from the mundane (like a lunch break!) to the very technical ones.
%Each possible reason for stoppage was represented by a symbol. Hence,
%the data given to us was in the form of very long sequences of symbols
%with time stamp. 
The aim was to extract interesting patterns from a sequence of logs
obtained from a manufacturing process by using machine learning techniques.
%The main question
%that our tool had to answer was: Does a particular sequence of breakdowns
%in one section of the manufacturing line influence a breakdown in some
%other section?
%My work was mainly implementation oriented and could be divided into two
%parts.  Firstly, 
I helped in the development of the data mining tools for this purpose.
%to filter the data 
%(like
%for example, removing the symbols corresponding to lunch break). This
%required using some scripting languages. Secondly, 
%I had to assist in
%enhancing our data mining tool by implementing new search techniques. The
%language of implementation was C.
%
%I was also involved in administrative activities related to the project and
%conducting exams.
%%
\section{\textnormal{MSc}}\label{sec:mastersAbstract}
(August 2001 to September 2003)
%This work is in the area of digital signal processing. 
We proposed a
framework for implementing wavelet transforms in the frequency domain
particularly for application on images and analyzed the performance of our
methods in the context of image compression by comparing our methods with
the methods used in JPEG and JPEG 2000 image compression
standards. 
%%
\section{\textnormal{Project associate}}\label{sec:turbo}
(October 2000 -- August 2001)
I worked in the area of communication theory. My work involved studying
the area of joint source and channel coding and other areas related to
telecommunication. I implemented some algorithms like turbo codecs and
Rijndael in MATLAB and helped in administrative activities.

\newpage
\section{Personnel Manager Information}
\noindent{\bf Madhav V. Marathe}

\noindent Director, Network Dynamics and Simulation Science Laboratory
and Professor, Computer Science\\
\\
\noindent{\bf Mailing Address:} \\
\noindent Network Dynamics and Simulation Science Laboratory \\
\noindent Virginia Bioinformatics Institute \\
\noindent Virginia Polytechnic Institute and State University \\
\noindent Research Building XV (0477) \\
\noindent 1880 Pratt Drive \\
\noindent Blacksburg, VA 24061 \\
\noindent USA \\
\\
{\bf Tel:} 540-231-8252\\
{\bf Fax:} 540-231-2891\\
{\bf Email:} marathe@vt.edu\\
{\bf Web page:}
http://www.vbi.vt.edu/ndssl/people/people-profile/Madhav-Marathe \\
\\
\\
\noindent{\bf L. Sunil Chandran}\\
Associate Professor,
Department of Computer Science and Automation
Indian Institute of Science\\
\\
\noindent{\bf Mailing Address:} \\
CSA Department\\
Indian Institute of Science\\
Bangalore - 560012\\
\\
{\bf Tel:} +91-80-22932368\\
{\bf Fax:} +91-80-23602911 \\
{\bf Email:} sunil@csa.iisc.ernet.in\\
{\bf Web page:} http://drona.csa.iisc.ernet.in/\textasciitilde sunil/


   }{}
\end{resume}
\bibliographystyle{mybib}
\nobibliography{mypapers}
\end{document}
