\documentclass{article}
\renewcommand\refname{} %for removing "References" heading in bib
\setlength{\topmargin}{0.25in}
\setlength{\columnsep}{2.0pc}
\setlength{\headheight}{0.0in}
\setlength{\headsep}{0.0in}
\setlength{\oddsidemargin}{-.19in}
\setlength{\evensidemargin}{-.19in}
\setlength{\parindent}{1pc}
\textheight 8.75in
\textwidth 6.8in
\newcommand{\ignore}[1]{}
%\bibliographystyle{plain}
\begin{document}
%\begin{center}
%Resume\vspace{.5cm}
%\end{center}
\noindent\textbf{ABHIJIN ADIGA} \\ 
\vspace{-3mm}\hrule\vspace{2mm}
\noindent 12900, Foxridge lane, Apt. D, \hfill Email:
abhijin@vbi.vt.edu, abhijin@gmail.com \\
Blacksburg, Virginia, 24060. \hfill Phone: +1 540 204 6679\\
\vspace{-2mm}\hrule\vspace{2mm}
\noindent\textbf{Current position: Postdoctoral Associate} (since October
2011) \\
Network Dynamics and Simulation Science Laboratory\\
Virginia Bioinformatics Institute\\
%Virginia Polytechnic Institute and State University\\
Research Building XV (0477), 1880 Pratt Drive\\
Blacksburg, VA 24061, USA\\
(\textbf{Note:} Job description is provided at the end.)\\
\vspace{-2mm}\hrule\vspace{-2mm}
\begin{center}
RESEARCH INTERESTS \\
\end{center}
\vspace{-2mm}\hrule\vspace{5mm}
My research area is graph theory and algorithms, with focus on:
\begin{itemize}
\item Approximation and parameterized algorithms
\item Graph dynamical systems and theoretical aspects of diffusion in
complex networks
\item Geometric representation of graphs: intersection graph
representations and dimensional properties of graphs
\item Algorithmic game theory
\end{itemize}
\vspace{0mm}\hrule\vspace{-2mm}
\begin{center}
EDUCATION \\
\end{center}
\vspace{-2mm}\hrule\vspace{5mm}
\textbf{PhD:} (August 2006--March 2011)\\
Department of Computer Science and Automation,\\
Indian Institute of Science\\ 
\\
{Supervisor}: {Dr. L Sunil Chandran}\\
{Thesis}: On dimensional parameters of graphs and posets\\
%~~~~\\
{Thesis Research Topics}: Geometric representation of graphs,
dimensional properties of graphs and posets. \\
~~~~~~~~~\\
%Courses (credited/attended):\hfill \textbf{CGPA: 7/8}\\
%Discrete Structures, Formal Methods in Computer Science,
%Algorithms in Coding, Topics in Graph Theory\\
\vspace{-2mm}\hrule\vspace{2mm}
\noindent\textbf{Master of Science (Engg):} (August 2001 -- August 2003)\\
Department of Electrical Engineering,\\
Indian Institute of Science,\\
\\
Supervisor: Dr. K R Ramakrishnan\\
Research Areas: Signal Processing, Image Processing, and Filter
Banks.\\
%Courses (credited/attended):\hfill \textbf{CGPA: 6.8/8}\\
%Linear algebra, Stochastic systems and applications, Information theory,
%Advanced digital signal processing, Digital signal compression, Modulation
%and coding theory.\\
%~~~~~~~~~\\
Thesis: ``Cyclic Multirate Structures Based on Symmetric-Periodic
Sequences.''\\
\vspace{-2mm}\hrule\vspace{2mm}
\noindent\textbf{Bachelor of Engineering:} (1996 - 2000)\\
Bangalore University (B.M.S. College of Engineering)\\
Discipline: Telecommunication Engineering\\
%Aggregate: 78\% (over 8 semesters)\\

\vspace{3mm}\hrule\vspace{-2mm}
\begin{center}
WORK EXPERIENCE \\
\end{center}
\vspace{-2mm}\hrule\vspace{2mm}
\begin{itemize}
\item
March 2011 -- September 2011\\
Research Associate, Dept. of Computer Science and Automation, IISc.\\
Area: Graph theory
\item
August 2004 -- July 2006\\
Worked in \textbf{Beceem Communications Pvt Ltd}. I was involved in the design and implementation of algorithms for
WiMax (802.16), a wireless communication standard. 
\item
October 2003 -- April 2004\\ 
Project Associate: worked under Dr. P S Sastry, Dept. of EE, IISc. \\ 
Area: Temporal Data-mining 
\item
October 2000 -- August 2001\\ 
Project Associate: worked under Dr. K R Ramakrishnan, Dept. of EE, IISc.\\  
Areas: Joint Source and Channel Coding,  Advanced Encryption Standard (Rijndael).
\end{itemize}
\vspace{2mm}\hrule\vspace{-2mm}
\begin{center}
ACHIEVEMENTS IN ACADEMICS\\
\end{center}
\vspace{-2mm}\hrule\vspace{2mm}
\begin{itemize}
\item Infosys Fellow (since January 2007): A fellowship awarded to select
PhD candidates in IISc by Infosys Technologies Ltd. 
\item Secured All India Rank of 34 in GATE 2000 (EC), a national level
entrance exam for post graduate studies.
\item Ranked 7th in Bangalore University in Telecommunication Engg. (Year 2000).
\end{itemize}

\hrule\vspace{-2mm}
\begin{center}
PUBLICATIONS\\
\end{center}
\vspace{-2mm}\hrule\vspace{2mm}
%\paragraph{\bf NOTE:} The publications marked T resulted out of the
%PhD thesis.
\begin{itemize}
\item \textbf{Lower bounds on boxicity}, A.~Adiga, L.~S.~Chandran and
N.~Sivadasan, accepted in Combinatorica.%\vspace{2mm}\\
%\begin{minipage}{.2cm}
%\mbox{}
%\end{minipage}
%\begin{minipage}{16cm}
%\it Although boxicity was introduced in 1969 and studied extensively,
%there are no significant results on lower bounds for boxicity. In this
%paper, we develop two general methods for deriving lower bounds. Both
%methods are based on vertex isoperimetric properties of the graph in
%consideration. Application of these methods have led to some significant
%results: (a) Almost all graphs have boxicity $\Omega(n)$.  (b) For a
%fixed $k$, boxicity of random $k$-regular graphs is $\Omega(k/\log k)$.
%\end{minipage}
%\end{itemize}
\item \textbf{How robust is the core of a network?},
A.~Adiga, A.~Kumar S. Vullikanti, accepted in
The European Conference on Machine Learning and Principles of Knowledge
Discovery in Databases (ECML PKDD'13), Prague, September 2013.
\item \textbf{Sensitivity of Diffusion Dynamics to Network Uncertainty},
A.~Adiga, C.~Kuhlman, H.~S.~Mortveit, A.~Kumar S. Vullikanti, accepted in
Proceedings of the Twenty-Seventh Conference on Artificial Intelligence
(AAAI'13), Bellevue, July 2013.
\item \textbf{Route Stability in Large-Scale Transportation Models},
A.~Adiga, H.~S.~Mortveit, S.~Wu, Workshop on Multiagent Interaction
Networks, AAMAS 2013.
\item \textbf{Cubicity, Degeneracy and Crossing Number}, A.~Adiga,
L.~S.~Chandran and R.~Mathew, accepted in European J. Combin. (Also
presented in FSTTCS 2011, 176-190).
\item \textbf{Parametrized and Approximation Algorithms for Boxicity},
A.~Adiga, J.~Babu and L.~S.~Chandran, accepted in IPEC-2012. 
\item \textbf{Representing a cubic graph as the intersection graph of
axis-parallel boxes in three dimensions}, A.~Adiga and L.~S.~Chandran,
accepted in Symposium on Computation Geometry (SoCG 2012).
\item \textbf{Boxicity and poset dimension}, A.~Adiga, D. Bhowmick and
L.~S.~Chandran, accepted in SIAM Journal on Discrete Mathematics (also
accepted in COCOON, Nha Trang, Vietnam, LNCS vol. 6196, 3-12,
2010).\ignore{\vspace{2mm}\\
\begin{minipage}{.2cm}
\mbox{}
\end{minipage}
\begin{minipage}{16cm}
\it In this paper we related the dimension of a poset $\mathcal{P}$ to the
boxicity of its underlying comparability graph $G_{\mathcal{P}}$. This
result has significant consequences for both parameters: (1) We showed
that $\mbox{dim}{\mathcal{P}}\le \mbox{\textnormal{treewidth}}(G_{\mathcal{P}})$. (2) It
is hard to approximate boxicity of a graph within a factor of
$\sqrt{n}$, where $n$ is the size of the graph. (3) We improved some bounds
for boxicity in terms of its maximum degree.
\end{minipage}}
\item \textbf{A constant factor approximation algorithm for boxicity of
circular arc graphs}, A.~Adiga, J.~Babu and L.~S.~Chandran, 
WADS, Brooklyn, USA, LNCS vol. 6844, 13-24, 2011. 
\item \textbf{Parameterized algorithms for boxicity}, A.~Adiga, R.~Chitnis
and S.~Saurabh, ISAAC, Jeju Island, Korea, LNCS vol. 6506, 366-377,
2010.\ignore{\vspace{2mm}\\
\begin{minipage}{.2cm} \mbox{}
\end{minipage}
\begin{minipage}{16cm}
\it Here, we initiated a study of structural parameterization of
boxicity. Our main result is that boxicity of a graph is fixed parameter
tractable when parameterized by vertex cover number. We have also given
approximation algorithms for boxicity when parameterized by feedback
vertex cover number and max leaf number.
\end{minipage}}
\item \textbf{The hardness of approximating the threshold dimension,
boxicity and cubicity of a graph}, A.~Adiga, D.~Bhowmick and L.~S.~Chandran,
Discrete Applied Mathematics, vol. 158 (16), 2010, 1719-1726.\ignore{\vspace{2mm}\\
\begin{minipage}{.2cm}
\mbox{}
\end{minipage}
\begin{minipage}{16cm}
\it This paper primarily relates two dimensional parameters: threshold dimension, which
is defined on graphs and dimension of a partially ordered set (poset).
We considered the class of split graphs and associated each graph in this
class with a poset, which we called the characteristic poset of the graph.
We showed that the threshold dimension of a split graph is of the
order of the dimension of its characteristic poset. This result has
the following algorithmic consequence: It is hard
to even approximate threshold dimension of a graph within a factor of
$\sqrt{n}$, where $n$ is the size of the graph.
\end{minipage}}
\item \textbf{Cubicity of interval graphs and
the claw number}, A.~Adiga and L.~S.~Chandran, Journal of Graph Theory 65(4):
323-333, 2010 (also presented in EuroComb 2009, Bordeaux, France).\ignore{\vspace{2mm}\\
\begin{minipage}{.2cm}
\mbox{}
\end{minipage}
\begin{minipage}{16cm}
\it We consider the following question: Given an interval graph $I$,
what is the minimum number of unit-interval graphs such that $I$ is
the edge intersection of these graphs, or in other words, what is the
cubicity of $I$? Suppose the largest induced star graph in $I$ has $k$
leaves. We show that cubicity of $I$ is bounded as follows: $\log_2(k)\le
\mbox{\textnormal{cub}}(I)\le \log_2(k)+2$.
\end{minipage}}
\item \textbf{Cubicity of threshold graphs}, A.~Adiga, Disc. Math.,
309(8): 2535-2537, 2009.\ignore{\vspace{2mm}\\
\begin{minipage}{.2cm}
\mbox{}
\end{minipage}
\begin{minipage}{16cm}
\it The result of this paper follows as a corollary of a result of the
above paper.
\end{minipage}}
\item \textbf{A design and implementation of orthonormal symmetric
wavelet transform using PRCC filter banks}, A.~Adiga, K.~R.~Ramakrishnan
and B.~S.~Adiga, ICASSP 2003, Hong Kong.\ignore{\vspace{2mm}\\
\begin{minipage}{.2cm}
\mbox{}
\end{minipage}
\begin{minipage}{16cm}
\it A framework for orthonormal symmetric wavelet transform is
proposed. The scheme consists of a design for cyclic wavelet transform
with real symmetric filters based on the Perfect Reconstruction Circular
Convolution Filter Banks. This is accompanied by an implementation in the Discrete
Trigonometric Transform domain. We also discuss briefly on
its performance in the context of applications such as image compression.
\end{minipage}}
\end{itemize}
%\begin{itemize}
%\item A. Adiga, L. S. Chandran, \emph{Cubicity of interval graphs and the claw
%number}, accepted in EuroComb'09, Bordeaux, France.
%\item A. Adiga, \emph{Cubicity of threshold graphs}, Discrete Mathematics,
%doi:10.1016/j.disc.2008.05.004.
%\end{itemize}
%\vspace{-5mm}
%\bibliography{docsdb}
%\noindent Under review:
%\begin{itemize}
%%\newpage
%\vspace{2mm}
%\hrule\vspace{-2mm}
%\vspace{2mm}\hrule\vspace{-1mm}
%\begin{center}
%PROGRAMMING SKILLS 
%\end{center}
%\vspace{-1mm}\hrule\vspace{2mm}
%Languages: C and MATLAB. \\
%Operating Systems: Linux and Windows.
%
%\vspace{2mm}\hrule
%\begin{center}
%\vspace{-1mm}REFERENCES \\
%\end{center}
%\vspace{-1mm}\hrule\vspace{2mm}
%\begin{itemize} 
%\item L. Sunil Chandran, Dept. of Computer Science and Automation,
%Indian Institute of Science, Bangalore (sunil@csa.iisc.ernet.in).
%\item Anil Vullikanti, Associate Professor, Department of Computer Science,
%and Virginia Bioinformatics Institute, Virginia Tech (akumar@vbi.vt.edu).
%\item K. R. Ramakrishnan, Dept. of Electrical Engineering, Indian
%Institute of Science, Bangalore \\ (krr@ee.iisc.ernet.in).
%\item Saket Saurabh, The Institute of Mathematical Sciences, Chennai
%(saket@imsc.res.in).
%\end{itemize}
%\vspace{2mm}\hrule
\vspace{2mm}\hrule
\begin{center}
\vspace{-2mm}COMMUNITY ACTIVITIES \\
\end{center}
\vspace{-2mm}\hrule\vspace{2mm}
Reveiwed papers for Graphs and Combinatorics, Information Processing
Letters and several conferences.

\vspace{2mm}\hrule
\begin{center}
\vspace{-2mm}REFERENCES \\
\end{center}
\vspace{-2mm}\hrule\vspace{2mm}
Available on request.

\newpage
\section{Current job (since October 2011)}\label{sec:currentJob}
As a post doctoral associate my work involves mathematical modeling
of complex graph dynamical systems which have diverse applications in
computer science, biology and sociology. Along with others in the group,
I focus on analysis and development of efficient algorithms for solving
combinatorial problems that arise out of the study of such systems.

\section{Research associate (March 2011 to October 2011)}
I continued my PhD work.

\section{PhD thesis abstract (August 2006 to March 2011)}
The work of this thesis falls into the broad area of theoretical computer
science, and in particular, concerns graph theory, combinatorics and
algorithms. We study some geometric representations of graphs and
dimensional properties of graphs and posets. The main objective is
to represent a graph using certain geometric objects. Many problems
on graphs which are otherwise computationally hard to solve can be
solved efficiently if the graph in consideration has a simple geometric
representation. In the parlance of theoretical computer science, many
NP-hard problems may be either solved in polynomial time or at least
have good approximation algorithms when a graph has a low dimensional
geometric representation. Besides, the representations which we consider
have applications in various other scientific disciplines: ecology,
social sciences, psychology, telecommunication and operations research.
Our contibution to this area are as follows:
(1) We have developed efficient algorithms to
obtain such representations for certain specific classes of graphs.
(2) We have complexity results which show that for some classes of graphs, it
is computationally hard to obtain such a representation.
(3) We give a general framework relating different representations.

\section{Beceem Communications Pvt. Ltd. (August 2004 -- July 2006)}\label{sec:beceem}
I was part of the algorithm development and simulation group. My
work involved the design and implementation of algorithms for WiMax
(802.16), a wireless broadband standard. I worked on (1) implementing
and maintaining the error control coding module in the simulator, 
%I had
%to implement the turbo encoder and decoder, a class of high-performance
%forward error correction (FEC) codes. The language in which the simulator
%was implemented was MATLAB.
(2) 
%One of the main objectives of the simulator was to serve as a reference
%for the performance of the actual WiMax chip, the product of the
%company. 
The creation of an interface between
the simulator (coded in MATLAB) and the chip (coded using Verilog), 
%I was involved in this project right from its initiation. This interface
%provided the vital link between the two modules and enabled the
%verification.
(3)
%Apart from these two major projects, I was also involved in implementing other parts of the simulator and 
Running simulations by
varying parameters such as type of codec, channel, etc.

\section{Work as project associate (October 2003 to April 2004)}\label{sec:dataMining}
The work was in the area of temporal data-mining. 
%We were given access
%to months of data from a manufacturing line of a factory. Each time
%the manufacturing process stopped, the time and the reason behind
%the stoppage was logged into the system. The reasons could be anything
%ranging from the mundane (like a lunch break!) to the very technical ones.
%Each possible reason for stoppage was represented by a symbol. Hence,
%the data given to us was in the form of very long sequences of symbols
%with time stamp. 
The aim was to extract interesting patterns from a sequence of logs
obtained from a manufacturing process by using machine learning techniques.
%The main question
%that our tool had to answer was: Does a particular sequence of breakdowns
%in one section of the manufacturing line influence a breakdown in some
%other section?
%My work was mainly implementation oriented and could be divided into two
%parts.  Firstly, 
I helped in the development of the data mining tools for this purpose.
%to filter the data 
%(like
%for example, removing the symbols corresponding to lunch break). This
%required using some scripting languages. Secondly, 
%I had to assist in
%enhancing our data mining tool by implementing new search techniques. The
%language of implementation was C.
%
%I was also involved in administrative activities related to the project and
%conducting exams.

\section{MSc thesis abstract (August 2001 to September 2003)}\label{sec:mastersAbstract}
%This work is in the area of digital signal processing. 
We proposed a
framework for implementing wavelet transforms in the frequency domain
particularly for application on images and analyzed the performance of our
methods in the context of image compression by comparing our methods with
the methods used in JPEG and JPEG 2000 image compression
standards. 


\section{Work as project associate (October 2000 -- August
2001)}\label{sec:turbo}
I worked in the area of communication theory. My work involved studying
the area of joint source and channel coding and other areas related to
telecommunication. I implemented some algorithms like turbo codecs and
Rijndael in MATLAB and helped in administrative activities.

\iffalse
\newpage
\section*{Personnel Manager Information}
\noindent{\bf Madhav V. Marathe}

\noindent Deputy Director, Network Dynamics and Simulation Science Laboratory
and Professor, Computer Science\\
\\
\noindent{\bf Mailing Address:} \\
\noindent Network Dynamics and Simulation Science Laboratory \\
\noindent Virginia Bioinformatics Institute \\
\noindent Virginia Polytechnic Institute and State University \\
\noindent Research Building XV (0477) \\
\noindent 1880 Pratt Drive \\
\noindent Blacksburg, VA 24061 \\
\noindent USA \\
\\
{\bf Tel:} 540-231-8252\\
{\bf Fax:} 540-231-2891\\
{\bf Email:} marathe@vt.edu\\
{\bf Web page:} http://ndssl.vbi.vt.edu/people/mmarathe.html\\
\\
\\
\noindent{\bf L. Sunil Chandran}\\
Associate Professor,
Department of Computer Science and Automation
Indian Institute of Science\\
\\
\noindent{\bf Mailing Address:} \\
CSA Department\\
Indian Institute of Science\\
Bangalore - 560012\\
\\
{\bf Tel:} +91-80-22932368\\
{\bf Fax:} +91-80-23602911 \\
{\bf Email:} sunil@csa.iisc.ernet.in\\
{\bf Web page:} http://drona.csa.iisc.ernet.in/\textasciitilde sunil/
\fi



\end{document}

