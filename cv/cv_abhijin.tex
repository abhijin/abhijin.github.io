\documentclass[margin,10pt]{res} % Use the res.cls style, the font size can be changed to 11pt or 12pt here
%
%%%%%%%%%%%%%%%%%%%%%%%%%%%%%%%%%%%%
%% Common preamble
%%%%%%%%%%%%%%%%%%%%%%%%%%%%%%%%%%%%
% PAGE
%% \usepackage{fullpage}
% FONTS
\usepackage{url} % enhanced version of computer modern
\urlstyle{same}
\usepackage{lmodern} % enhanced version of computer modern
\usepackage[T1]{fontenc} % for hyphenated characters
\usepackage{amssymb}
\usepackage{mathtools} % contains amsmath which comes with align
\usepackage{amsthm}
\usepackage{microtype} % some compression
%%
%% REFERENCING DOES NOT WORK WITH BIBENTRY PACKAGE
\usepackage{natbib}
\usepackage[colorlinks=true,pdfborder={0 0
0},citecolor=magenta,linkcolor=magenta]{hyperref}
\newcommand{\tuta}{\emph{T.~absoluta}}
\newcommand{\tutafull}{\emph{Tuta~absoluta}}

\usepackage{bibentry}
% LISTS
\usepackage{enumerate}
% ROMAN NUMERALS
\makeatletter
\newcommand{\rmnum}[1]{\romannumeral #1}
\newcommand{\Rmnum}[1]{\expandafter\@slowromancap\romannumeral #1@}
\makeatother
%%%%%%%%%%%%%%%%%%%%%%%%%%%%%%%%%%%%
%% preamble ends
%% from now on, draft specific
%%%%%%%%%%%%%%%%%%%%%%%%%%%%%%%%%%%%
\setlength{\textwidth}{5.1in} % Text width of the document
\usepackage{etoolbox}
\usepackage{multicol}
%% adding footer
\usepackage{fancyhdr}
\usepackage{xcolor}
%%\fancypagestyle{mypagestyle}{%
%%\fancyhf{}%
%%\fancyfoot[C]{\textcolor{red}{\thepage}}%
%%}
\rfoot{\textcolor{gray}{Updated on \today}}
\fancyhead{}
\renewcommand{\headrulewidth}{0pt}
\pagenumbering{gobble}
\newtoggle{compact}
%% \toggletrue{compact} % comment this if you want normal resume

\begin{document}
\nobibliography*
\thispagestyle{fancy}
%----------------------------------------------------------------------------------------
%	NAME AND ADDRESS SECTION
%----------------------------------------------------------------------------------------

%\moveleft.5\hoffset\centerline{\large\bf Abhijin Adiga\hfill} % Your name at the top
\iftoggle{compact}{
    \moveleft1.0\hoffset\centerline{\large\bf Abhijin Adiga (Short
    r\'{e}sum\'{e})\hfill} % Your name at the top
}{
\moveleft1.0\hoffset\centerline{\large\bf Abhijin Adiga\hfill} % Your name at the top
}
 
\moveleft\hoffset\vbox{\hrule width\resumewidth height 1pt}\smallskip % Horizontal line after name; adjust line thickness by changing the '1pt'
 
%----------------------------------------------------------------------------------------

\begin{resume}

%----------------------------------------------------------------------------------------
%	OBJECTIVE SECTION
%----------------------------------------------------------------------------------------
%%  
%% \section{OBJECTIVE}  
%% 
%% A position in the field of computers with special interests in business applications programming, information processing, and management systems. 
%% 
%----------------------------------------------------------------------------------------
%	EDUCATION SECTION
%----------------------------------------------------------------------------------------
\section{\textnormal{\textsc{Contact Information}}}
Network Systems Science and Advanced Computing (NSSAC)\\
Biocomplexity Institute and Initiative\hfill \texttt{email:}
abhijin@virginia.edu\hspace{0mm}\mbox{}\\
University of Virginia\hfill \texttt{phone:}
\texttt{+}1 540 204 6679\hspace{6.5mm}\mbox{}\\
\url{https://abhijin.github.io}
%%
\section{\textnormal{\textsc{Current\,\,\,\, Position}}}
\textbf{Research Associate Professor} \hfill 2022 -- \\
Network Systems Science and Advanced Computing (NSSAC)\\
University of Virginia
%%%%
\section{\textnormal{\textsc{Research interests and Focus}}}
Broadly, my interests are in network science, data science, algorithm
design, combinatorics, machine learning and game theory. My current focus
is on studying diffusion processes using network dynamical systems and
machine learning with application to the spread of invasive species,
infectious diseases, and other socio-technical phenomena. There are three
main thrusts to my research.  Through large collaborative efforts, I design
and analyze data- and process-driven models using domain-specific
knowledge. Secondly, I strive to advance the theoretical foundations of
such models to create robust models that can be applied in data-poor
environments. Thirdly, I develop efficient practical solutions with
provable guarantees for various computational problems that arise in the
target applications.\\


%%%%
\iftoggle{compact}{}{
%%%%%%
\section{\textnormal{\textsc{Work\,\,\,\, Experience}}} 
Research Assistant Professor \hfill 2018 -- 2022\\
Network Systems Science and Advanced Computing (NSSAC)\\
University of Virginia\\
~~~\\
{Research Assistant Professor} \hfill Jul 2016 -- Oct 2018\\
Senior Research Associate \hfill May 2014 -- Jul 2016\\
Postdoctoral Associate \hfill October 2011 -- May 2014\\
Network Dynamics and Simulation Science Laboratory\\
Biocomplexity Institute of Virginia Tech\\
~~~\\
Research Associate \hfill March 2011 -- September 2011\\
Dept. of Computer Science and Automation, IISc \\
~~~\\
{Beceem Communications Pvt Ltd}\hfill August 2004 -- July 2006\\
Algorithm design for WiMax (802.16) \\
~~~\\
Project Associate \hfill October 2003 -- April 2004\\ 
Project Associate\hfill October 2000 -- August 2001\\ 
Dept. of Electrical Engineering, IISc \\
%% Worked under Dr. P S Sastry on temporal data-mining \\
%% Worked under Dr. K R Ramakrishnan on source \& channel coding and Advanced
%% Encryption Standard (Rijndael)
}

\section{\textnormal{\textsc{Education}}}
\textbf{PhD:} \hfill August 2006 -- March 2011\\
Dept. of Computer Science and Automation\\
Indian Institute of Science, Bangalore, India\\ 
\iftoggle{compact}{}{
%% Adviser: {Dr. L Sunil Chandran}\\
%% {Thesis}: On Dimensional Parameters of Graphs and Posets\\
~\\
\noindent\textbf{Master of Science (Engg):} \hfill August 2001 -- August 2003\\
Dept. of Electrical Engineering,\\
Indian Institute of Science, Bangalore, India\\
%% Adviser: Dr. K R Ramakrishnan\\
%% {Thesis:} Cyclic Multirate Structures Based on Symmetric-Periodic
%% Sequences\\
~\\
\noindent\textbf{Bachelor of Engineering:} \hfill 1996 -- 2000\\
Bangalore University (B.M.S. College of Engineering)\\
Telecommunication Engineering
}
%%%%%%
\iftoggle{compact}{}{
\section{\textnormal{\textsc{Grants}}}
\begin{enumerate}[$\circ$]
\item USDA NIFA Foundational and Applied Science Program: \emph{Network
Models of Food Systems and their Application to Invasive Species Spread},
Amount:~\$400,000; Duration: Sep'19--Aug'23,
Role: PI
\item USAID IPM Innovation Labs: \emph{Assessment of Invasive Alien Species
Distribution in the Chitwan-Annapurna-Landscape (CHAL) Region, Nepal},
Amount:~\$150,000; Duration: Jan'19--Nov'21,
Role: Co-PI
\item USAID Egypt Mission: \emph{Pest Risk Assessment of the Fall Armyworm,
Spodoptera frugiperda in Egypt}, Amount:~\$18,000; Duration:
Oct'17--Dec'17, Role: Co PI
%\item
%\noindent\textbf{Pending}\\
\item USAID IPM Innovation Labs: \emph{A High-resolution Interaction Based
Approach to Modeling the Spread of Agricultural Invasive Species},
Amount:~\$1,000,000 (\$800,000 for Virginia Tech); Duration:
Oct'15--Nov'21,
Role: PI
\end{enumerate}

\noindent
\textbf{Recent grants:}\\
\begin{enumerate}[$\circ$]
\item Management of Current and Emerging biotic and abiotic stress threats
to rice and vegetable crops and the environment of Madhesh Province, Nepal.
Proposal submitted Einstein project. October. 2023
%% \item USAID Feed the Future Innovation Lab for Current and Emerging Threats
%% to Crops Catalog of Federal Domestic Assistance, 2021.
%% \item NSF HDR Institute: Data Science for Transportation, Epidemiology and
%% Power (DataSTEP), 2021 (submitted).
%% \item NSF AI Institute: Agricultural AI for Transforming
%% Workforce and Decision Support (AgAID), 2021.
\end{enumerate}
}

%%%%%%%%%%%%
\iftoggle{compact}{
\section{\textnormal{\textsc{Selected awards}}}
}{
\section{\textnormal{\textsc{Awards}}}
}
\begin{enumerate}[$\circ$]
    \item IJCAI-22 Distinguished PC Member (top 3\%).
    \item IJCAI-21 Distinguished SPC Member.
\iftoggle{compact}{}{
   \item ``DSFEW Early Career Researchers Travel Fund'', KDD 2016.
   }
\item ``Honorable Mention For Outstanding Novelty of Research
Question'' award for the paper ``Sensitivity of Diffusion Dynamics to
Network Uncertainty'' in AAAI'13.
\iftoggle{compact}{}{
\item Infosys Fellow: awarded to select
PhD candidates in IISc by Infosys Technologies Ltd. 
\item Secured All India Rank of 34 in GATE 2000 (EC), a national level
entrance exam for post graduate studies.
\item Ranked 7th in Bangalore University in Telecommunication Engg. (Year 2000).
   }
\end{enumerate}
%%
%%
%% \iftoggle{compact}{}{
%% \section{\textnormal{\textsc{News}}}
%% \begin{enumerate}[$\circ$]
%%     \item 
%%     \href{https://ipmil.cired.vt.edu/wp-content/uploads/2021/09/Hendery_BES_TheNiche_523_Autumn_2021.pdf}{Monitoring
%%     and managing the tomato leafminer}, The Niche, Autumn 2021.
%% \item 
%% \href{https://knowablemagazine.org/article/technology/2020/virtual-agents-change-how-computers-are-mapping-covid19s-future}{Virtual
%% agents of change: How computers are mapping Covid-19’s future}, Knowable
%% Magazine, 2020.
%% \item
%% \href{https://www.agrilinks.org/post/q-why-mapping-spread-invasive-species-above-so-important-prevention-efforts-below}{Agrilinks
%% article on USAID Invasive Species project; 2020.}
%% \item
%% \href{https://www.cbs19news.com/story/41855331/grant-to-help-model-potential-arrival-of-invasive-species-and-protect-tomato-crops?utm_source=cmpgn_news&utm_medium=email&utm_campaign=vtUnirelNewsDailyCMP_031320-f%2Fs}{Charlottesville
%% News (CBS19) announcing the USDA FACT project; 2019.}
%% \item
%% \href{https://vtnews.vt.edu/articles/2018/03/outreach-ipmegyptfaw.html}{Virginia
%% Tech provides key intel in U.S. and Egyptian-led battle against a major
%% pest; 2018}
%% \item
%% \href{https://vtnews.vt.edu/articles/2018/02/outreach-ipmnexus.html}{Countries
%% get heads up about leafminer invasion thanks to Virginia Tech} (also picked
%% up by Wisconsin Farmer and Agrilinks)
%% \item
%% \href{https://vtnews.vt.edu/articles/2016/05/outreach-oiredinnovationlabawards.html}{Virginia Tech awards more than \$11 million to help feed people in
%% developing countries}
%% \item
%% \href{https://www.bi.vt.edu/ndssl/news/virginia-tech-research-team-fights-the-spread-of-invasive-pests}{Virginia
%% Tech Research Team Fights the Spread of Invasive Pests}
%% \end{enumerate}
%% }

\iftoggle{compact}{
\section{\textnormal{\textsc{Publications count}}}
Journals: 30\\
Conference proceedings: 46\\
Workshops/posters/abstract: 14\\

\section{\textnormal{\textsc{Grants summary}}}
PI: USAID \$1,100,000; USDA \$400,000\\
Co-PI: Two USAID grants \$170,000
}{}

\iftoggle{compact}{}{
\section{\textnormal{\textsc{Impact}}}
\begin{enumerate}[$\circ$]
\item USAID and USDA invasive species projects featured in Agrilinks,
Wisconsin Farmer, Charlottesville News, Knowable Magazine, and Virginia
Tech news.
\item USAID project mentioned in the USAID Center for Emerging Threats of
Crops Notice of Funding Opportunity as a project that has informed the US
of emerging threats and helped prepare for impending invasion.
\item Supported BII's COVID-19 response efforts in (i)~network construction,
analysis, and validation and (ii)~simulation analytics. 
\item Invited talks in Egypt and Indian Council of Agricultural Research,
India. Talks in various conferences in national and international venues on
invasive species modeling: Ethiopia, India, International Congress of
Entomology (Orlando), Symposium on Integrated Pest Management (Washington
DC).
\item Webinar: New Approaches to Control the South American Tomato Leaf
Miner \emph{Tuta absoluta}, April 2018
\end{enumerate}
}

\section{\textnormal{\textsc{Mentorship}}}
\begin{enumerate}[$\circ$]
    \item Co-advisor for two Masters student (Thesis accepted) and one PhD
    student.
    \item Part of thesis committee for three PhD students. I advised on
    several parts of the thesis of two of them.
    \item Research assistants (PhD 1 and Masters students 4)
    \item Undergraduates: 9
    \item High-school: 1
\end{enumerate}

\section{\textnormal{\textsc{Professional Service}}}
\begin{enumerate}[$\circ$]
    \iftoggle{compact}{}{
    \item Professional service: 
    \begin{itemize}
    }
        \item Grant review panel (NSF~1, USDA~5, FONDECYT Chile~1)
        \item Senior TPC (3) and TPC (10)
        \item Reviewed papers for a number of journals and conferences
        spanning multiple domains such as Nature comm. Bio., Journal of
        Pest Science, JPDC, ACM Trans. on Algo, Journal of Royal Soc.
        Interface, etc.
        \iftoggle{compact}{}{
    \end{itemize}
    \item Biocomplexity Institute \& Initiative:
    \begin{itemize}
        \item Student and postdoc hiring committee in NSSAC 2018--2020.
        \item Student and postdoc hiring committee in NDSSL 2017--2018.
\item Member of graduate students admission team in NDSSL for the Fall'16
admissions
    \end{itemize}
}
\end{enumerate}

\iftoggle{compact}{}{
\section{\textnormal{\textsc{Programmatic contributions}}}
\begin{enumerate}[$\circ$]
\item As PI and Co-PI of the USAID and USDA projects, I have led the
research in the area of invasive species modeling. This includes
collaborating with people from multiple domains and countries (France,
Senegal, India, Nepal, Bangladesh, and US), presenting in annual meetings,
data exploration, providing content for news reports, mentoring students,
and preparing annual reports. It has resulted in publications in top venues
including Proceedings of the Royal Society Biology, Journal of Pest
Science, Journal of Crop Protection and IEEE BigData conference.
\item COVID-19 response: I led the modeling and development of certain
modules in the synthetic population generation pipeline. This work has
contributed to the generation of US domestic networks as well as global
networks. Also, I developed a network analysis tool set, which is used to
validate our networks and for comparative analysis. These tools have been
applied (i)~to provide weekly inputs to various agencies during certain
periods of the year 2020-21, and (ii)~in manuscripts submitted/under
preparation.
\item DARPA NGS2: I contributed significantly to the theoretical aspects of
this project. Our work on inferring graphical dynamical systems has
resulted in five publications in top AI venues and several workshop
presentations.
\item Simulation analytics framework: \textcolor{red}{PENDING}
\item Fall armyworm in Egypt: In a collaborative effort, I led the modeling
effort to assess the possible spread of Fall armyworm in Egypt. This was
funded by USAID mission in Egypt.
\item Participated in a number of proposal writing efforts every year.
\end{enumerate}

\section{\textnormal{\textsc{Software and Datasets}}}
\begin{enumerate}[$\circ$]
\item Multi-pathway simulator: I have led the development of a simulation
framework to study the multi-pathway spread of invasive species. It
consists of a simulator of a generic network diffusion process implemented
using vectorized methods in Python, a multi-scale temporal network module,
implementation of algorithms for calibration and interventions, modules for
model space exploration using computing clusters, regression tests, and various
visualization tools. Domestic trade networks have been constructed for
several countries using multiple datasets and expert knowledge. The
resulting simulation framework has been applied in multiple studies. The
simulator and synthetic datasets are publicly available and are constantly
updated.
\item High-resolution synthetic population models and datasets: Our group
(NSSAC) has been synthesizing highly-detailed population models from
multiple data sources for more than~15 years. Over the years, these
datasets have been applied in epidemiology (COVID-19, Ebola, influenza,
malaria, etc.), transportation, disaster preparedness, resilience and
sustainability. I have played a prominent part in the Biocomplexity
Institute's COVID-19 response on the modeling and development of the
synthetic population networks. I co-led the design and implementation of
the physical contact network construction module. I applied concepts from
geometric intersection graphs and parallelization to speed up the network
generation. I have also developed a tool for analyzing the generated
networks. It has been regularly applied to compare different networks,
visualize, and validate our models. This software has been well integrated
in to our synthetic population pipeline.
\item Deep learning and remote sensing: We have developed a convolutional
neural network (CNN) framework to predict the distribution of invasive
plants using multispectral satellite images and field survey data.  Our
robust training and evaluation framework employs multiple hold-out approach
for model selection and transfer learning to cope with data challenges
imposed by field survey and imagery constraints.  Multi-spectral
remote-sensed images from multiple satellites were used in this study. We
have developed the framework for optical calibration, sharpening, and
interpolation of the images towards feature vector extraction. Popular
deep neural networks had to be adapted for satellite images. Transfer
learning approaches were applied. The framework is applied to study the
distribution of multiple invasive plants in the Chitwan-Annapurna Landscape
of Nepal, a biodiversity hotspot.
\end{enumerate}

\section{\textnormal{\textsc{Transdisciplinary team science}}}
My work involves leading and being part of large teams of researchers from
different fields.  As PI of two USAID and USDA projects, I have led BII’s
research in the area of invasive species modeling. As PI, I have initiated
multiple projects collaborating with researchers from several countries
(US, France, Senegal, India, Nepal, and Bangladesh). Example projects
include (i)~modeling the spread of a pest of the tomato plant, \tutafull{} in
Southeast Asia and West Africa involving entomologists, economists,
modelers and computer scientists and (ii)~mapping invasive plants in Nepal
using remote-sensing and machine learning involving botanists and
geoinformation specialists. I play a major role is several large team projects in the
Biocomplexity Institute as well. These include studies related to
computational epidemiology such as COVID-19 response, disaster
preparedness, and computational social science.

%%
\section{\textnormal{\textsc{Talks}}} 
\begin{enumerate}[1.]
\item US-Scale High Resolution Digital Twin of Coupled Livestock, Wild
Birds, and Human Populations Ecosystem, CDC (six presentations), September
to December 2024.
\item US-Scale High Resolution Digital Twin of Coupled Livestock, Wild
Birds, and Human Populations Ecosystem, Spillover from Highly Pathogenic
Avian Influenza, LANL, National Press Club, September 2024.
\item Identifying Complicated Contagion Scenarios from Cascade Data, 29th
ACM SIGKDD Conference on Knowledge Discovery and Data Mining, August
2023.
\item A Robust Deep Learning Framework Reveals the Spread of Multiple
Invasive Plants in a Biodiversity Hotspot using Satellite Imagery, The
Workshop on Artificial Intelligence for Social Good (in AAAI'23), February
2023.
\item How to Stop an Epidemic? Network Dynamics and Simulation Systems,
CheckedIt, India (virtual), February 2022.
\item (\textbf{Invited}) Learning the Local and Global Behavior of
Dynamical Systems on Networks, Workshop on AI Socio-\'{e}cosysteme et
r\'{e}silience, Montpellier, France (virtual), November 2021.
\item Boolean Games: Inferring Agents' Goals Using Taxation Queries,
International Joint Conference on Artificial Intelligence (IJCAI'20)
(virtual), January 2021.
\item A Deep Learning Framework for Invasive Species Mapping using
High-Resolution Satellite Imagery, ASPRS 2020 Annual Conference (virtual), June 2020.
\item (\textbf{Invited}) Network Dynamical Systems: Theory and
Applications, Indian Institute of Technology, Hyderabad, India,
November 2019.
\item Modern AI Techniques to Understand the Spatio-temporal Spread of
Invasive Alien Plants: Approaches and Challenges, International Plant
Protection Congress, Hyderabad, India, November 2019.
\item Modeling the multi-pathway spread of agricultural pests using
network science, International Plant Protection Congress, Hyderabad,
India, November 2019.
\item Understanding the Role of Seasonal Food Trade Networks in Invasive
Species Spread, SIAM Network Science, Snowbird, Utah,
May 2019.
\item (\textbf{Invited}) How to stop an epidemic? Networked dynamical
systems, games and near-optimal algorithms, Indian Institute of
Technology, Dharwad, October 2018.
\item (\textbf{Invited}) Multi-pathway models to assess the threat of
invasive species spread, Indian Agricultural Research Institute, Delhi,
October 2018.
\item Multi-pathway models to understand the spread and impact of
\emph{Tuta
absoluta}, International Conference on Biological Control (ICBC),
September 2018.
\item (\textbf{Webinar}) New Approaches to Control the South American Tomato
Leaf Miner \emph{Tuta absoluta}, April 2018
\item Monitoring the spread of {\it Tuta absoluta} using a
multi-layered network based modeling framework, \emph{9th International
IPM Symposium}, Baltimore, March 2018
    \item (\textbf{Invited}) Modeling the Spread of Fall Armyworm,
    \emph{Fall Armyworm Workshop}, Adis Ababa, 2017
    \item (\textbf{Invited}) Understanding the role of human-mediated
    pathways in pest spread: Case study of \emph{Tuta absoluta}, \emph{12th
    Arab Congress of Plant Protection}, Hurghada, 2017
\item Monitoring spread of T. absoluta using a multi-layered network
based modeling framework, \emph{Symposium on Global Spread and Management
of the South American Tomato Leafminer, Tuta absoluta. International
Congress of Entomology}, Orlando, 2016
\item (\textbf{Invited}) How to stop an epidemic?  Games and near-optimal
algorithms, \emph{Dept. of Computer Science and Automation, Indian
Institute of Science}, Bangalore, 2014
\item (\textbf{Invited}) Sensitivity of Dynamical Properties to Network
Uncertainty, \emph{Dept. of Computer Science and Automation, Indian
Institute of Science}, Bangalore, 2013
\end{enumerate}
}

\iftoggle{compact}{
\section{\textnormal{\textsc{Selected publications}}}
\begin{enumerate}[1.]%%[$\circ$]
\item \bibentry{adiga2022network}
\item \bibentry{poudel2020predicting}
\item \bibentry{mcnitt2019assessing}
\item \bibentry{maharjan2019predicting}
\item \bibentry{venkatramanan2020modeling}
\item \bibentry{campos2017western}
\item \bibentry{zhang2016near}
\item \bibentry{mishra2023reconstructing}
\item \bibentry{adiga2021realistic}
\item \bibentry{adiga2014sensitivity}
\end{enumerate}
}{
\section{\textnormal{\textsc{Publications}}} 
\noindent\textbf{Journal articles}\\
\begin{enumerate}[1.]%%[$\circ$]
\item \bibentry{bhattacharya2024novel}
\item \bibentry{chen2024role}
\item \bibentry{adiga2022using}
\item \bibentry{adiga2022network}
\item \bibentry{cedeno2020networked}
\item \bibentry{poudel2020predicting}
\item \bibentry{adiga2022using}
\item \bibentry{adiga2022network}
\item \bibentry{de2021thermal}
\item \bibentry{mcnitt2019assessing}
\item \bibentry{maharjan2019predicting}
\item \bibentry{venkatramanan2020modeling}
\item \bibentry{adiga2019graphical}
\item \bibentry{adiga2018disparities}
\item \bibentry{adiga2018sublinear}
\item \bibentry{campos2017western}
\item \bibentry{adiga2017inferring}
\item \bibentry{adiga2017activity}
\item \bibentry{zhang2016near}
\item \bibentry{adiga2014sensitivity}
\item \bibentry{adiga2014representing}
\item \bibentry{adiga2014constant}
\item \bibentry{wu2014limit}
\item \bibentry{adiga2014lower}
\item \bibentry{adiga2014cubicity}
\item \bibentry{adiga2011boxicity}
\item \bibentry{adiga2010hardness}
\item \bibentry{adiga2010cubicity}
\item \bibentry{adiga2009cubicityThreshold}
\item \bibentry{saha2015spectrum}
\item \bibentry{adiga2013robust}
\end{enumerate}
\clearpage
\noindent\textbf{Refereed conference proceedings}\\
\begin{enumerate}[1.]%%[$\circ$]
\item \bibentry{qiu2024learning}
\item \bibentry{adiga2024value}
\item \bibentry{aljundi2023network}
\item \bibentry{tanvir2023machine}
\item \bibentry{harrison2023identifying}
\item \bibentry{trabelsi2023resource}
\item \bibentry{mishra2023reconstructing}
\item \bibentry{trabelsi2023resource}
\item \bibentry{mishra2022community}
\item \bibentry{trabelsi2022resource}
\item \bibentry{rosenkrantz2022efficiently}
\item \bibentry{chen2022effective}
\item \bibentry{trabelsi2022maximizing}
\item \bibentry{adiga2021realistic}
\item \bibentry{bhattacharya2021ai}
\item \bibentry{hoops2021high}
\item \bibentry{adiga2020booleangames}
\item \bibentry{adiga2020bounds}
\item \bibentry{adiga2019validating}
\item \bibentry{hu2019modeling}
\item \bibentry{adiga2019pac}
\item \bibentry{cedeno2019mechanistic}
\item \bibentry{ren2018generative}
\item \bibentry{adiga2018inferringprob}
\item \bibentry{adiga2018learning}
\item \bibentry{adiga2017kserver}
\item \bibentry{venkat2017towards}
\item \bibentry{adiga2016delay}
\item \bibentry{adiga2016temporal}
\item \bibentry{adiga2015csd}
\item \bibentry{zhang2015group}
\item \bibentry{adiga2015inferring}
\item \bibentry{adiga2015network}
\item \bibentry{adiga2015effect}
\item \bibentry{saha2015spectrum}
\item \bibentry{saha2014game}
\item \bibentry{adiga2013modelingurban}
\item \bibentry{adiga2013subgraph}
\item \bibentry{adiga2013robust}
\item \bibentry{adiga2013sensitivity}
\item \bibentry{adiga2012representing}
\item \bibentry{adiga2012polynomial}
\item \bibentry{adiga2011cubicity}
\item \bibentry{adiga2011constant}
\item \bibentry{adiga2010parameterized}
\item \bibentry{adiga2010boxicity}
\item \bibentry{adiga2009cubicity}
\item \bibentry{adiga2003design}
\end{enumerate}
\noindent\textbf{Workshop/Extended abstract}\\
\begin{enumerate}[1.]
\item \bibentry{bintey2024irrnet}
\item \bibentry{adiga2023robust}
\item \bibentry{trabelsi2022maximizing}
\item \bibentry{mishra2022communities}
\item \bibentry{adiga2020boolean}
\item \bibentry{fox2019learning}
\item \bibentry{adiga2019inferring}
\item \bibentry{adiga2019understanding}
\item \bibentry{nath2018using}
\item \bibentry{adiga2018using}
\item \bibentry{adiga2018inferring}
\item \bibentry{geneiusWu}
\item \bibentry{hybridModelsVenkat}
\item \bibentry{climexSridhar}
\item \bibentry{venkat2016kdd}
\item \bibentry{adiga2015bme}
\item \bibentry{adiga2015synthetic}
\item \bibentry{adiga2012route}
\end{enumerate}
\noindent\textbf{Reports}\\
\begin{enumerate}[1.]
\item \bibentry{heinrichs2018pest}
\end{enumerate}
%% \noindent\textbf{Under review/Technical reports}\\
%% \begin{enumerate}[1.]
%% \item \bibentry{mortveit2015synthetic}
%% \item \bibentry{adiga2015sublinear}
%% \item \bibentry{adiga2015interaction}
%% \end{enumerate}
%%
\section{\textnormal{\textsc{Students \\Current/Past}}}
\noindent\textbf{PhD} (GRA)\\
Ritwick Mishra (Fall'21--) (Co-adviser with Anil
Vullikanti for masters thesis),
Manisha Sudhir (Spring'20--Spring'21) (Co-adviser with Anil
Vullikanti),
Amro Aljundi (Spring'23--) (Adviser: Madhav Marathe),
Galen Harrison (Spring'23--) (Adviser: Madhav Marathe),
Rituparna Datta (Spring'23--) (Adviser: Anil Vullikanti),
Prathyush Sambaturu (Thesis committee),
Sichao Wu (Thesis adviser: Henning Mortveit)\vspace{.2cm}\\
\noindent\textbf{Masters} (GRA)\\
Hongze Chen (Spring'24)\\
Sanchit Sinha (Spring'21)\\
Aniruddha Dave (Fall'20)\\
Daniel Perez Lazarte (Fall'19, Spring'20)\\
Joseph McNitt (Thesis adviser: Henning Mortveit)\vspace{.2cm}\\
\noindent\textbf{Undergraduates}\\
Alex Fetea (Summer'23--Fall'24) (Co-adviser Samarth Swarup)
Alexander Yao (Summer'24) (Co-adviser Samarth Swarup)
Chris Goodhart (Summer'23) (Co-adviser Samarth Swarup),
Andrew Ma (Summer'23) (Co-adviser Samarth Swarup),
William Mueller (Summer'22, Fall'22),
Clark Mollencop (Summer'22),
Neha Pattanaik (Summer'21),
Penina Waghalter (Summer'21),
Nicholas Palmer (Summer'21),
Johnny Yang (Fall'20, Spring'21),
Surbhi Singh (Fall'19--Spring'20),
Ethan Choo (Summer'19),
Katie Liu (Summer'19),
Bryan Kaperick (Spring'16--Spring'17), and
Amleshwar Kumar (Intern: Fall'16) \vspace{.2cm}\\
\noindent\textbf{High school}\\
Alexis Fox (Fall'23--Sprint'24) (Co-adviser Samarth Swarup) \\
Manu Amundsen (Spring'21)\vspace{.2cm}\\
\noindent\textbf{Student thesis/project committee}\\
Tanay Mehta (PhD, Northeastern University),
Sudip Saha (PhD, Virginia Tech)
%%\section{\textnormal{Memberships}}
%%\begin{enumerate}[$\circ$]
%%\item IEEE member
%%\end{enumerate}
%%
%%
\section{\textnormal{\textsc{Professional Service}}}
\noindent\textbf{Guest editor}\\
Journal of Indian Institute of Science (2021)
\smallskip\\
\noindent\textbf{Senior Technical Program Committee member}\\ 
IJCAI~(2021--2023)
\smallskip\\
\noindent\textbf{Technical Program Committee member}\\ 
IJCAI~(2024,2025), IJCAI AI and Social Good~(2025), KDD~(2025), KDD Data
Track~(2025), AAMAS~(2025), ICLR~(2025), AAAI~(2021--2025), HiPC~(2024),
BigData~(2024), ANNSIM~(2021), AIKE~(2018--2021), PhD-ASONAM (2020),
INFOCOM~(2019), CSoNet~(2016), CONECCT~(2015), SDM-Networks~(2015),
SIAMNS~(2015)\smallskip\\
\noindent\textbf{Grant Review}\\
USDA~(Fall'24, Fall'23, Spring'23, Fall'22, Fall'21, Fall'20 and Spring'20) (Grant review panelist)\\
NSF~(2018) (Grant review panelist)\\
National Fund for Scientific and Technological Development (FONDECYT),
Chile\smallskip\\
\noindent\textbf{Reviewer}\\ 
\noindent
NITK thesis~(2025),
SODA~(2025),
ICML~(2024),
Plos~Comp.~Bio.~(2023),
AAAI~(2023),
SNAM~(2022),
Entomologia Generalis (2022),
Entomologia Generalis (2022),
PlosOne (2022),
ICML (2022 multiple papers),
Biological Control (2022),
Applied Network Science (2019--2022),
Nature Comm. Biology (2021),
Journal of Pest Science~(2020, 2018), 
WG~(2020),
International Journal of epidemiology~(2019),
Pest Management Science~(2019), 
Journal of Parallel and Distributed Computing~(2019), 
Australasian Journal of Combinatorics~(2018, 2015),
FPSAC~(2017),
ACM Transactions on Algorithms~(2017),
Journal of Royal Society Interface~(2017), INFOCOM~(2016, 2015), Order~(2015), 
Algorithmica~(2014), Journal of Autonomous Agents and Multi-Agent
Systems~(2013), 
Information Processing Letters~(2012), 
Graphs and
Combinatorics~(2011), CATS~(2011)\smallskip\\
\noindent\textbf{Advisory Committee}\\
E\textsuperscript{2}JDJ\bigskip\\
\noindent\textbf{Miscellaneous}\\ 
\begin{enumerate}[$\circ$]
\item Student and postdoc hiring committee in NSSAC 2018-2021
\item Student and postdoc hiring committee in NDSSL 2017-2018
\item Member of graduate students admission team in NDSSL for the Fall'16
admissions
\item Organized NDSSL graduate seminar series for the academic year 2013-2014
\end{enumerate}
}
%%
\end{resume}
\bibliographystyle{abbrv}
\nobibliography{mypapers}
\end{document}
